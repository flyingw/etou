\documentclass[twoside]{article}
\usepackage[paperwidth=13cm,paperheight=20cm,top=2cm,bottom=2cm,left=1.2cm,right=1.2cm,columnsep=2cm]{geometry}
\setcounter{secnumdepth}{0}
\setlength{\parindent}{3mm}

\usepackage{specific}

% fonts
\usepackage[english,ukrainian]{babel}
\usepackage{fontspec}
\setmainfont{Libertinus Serif}
\setsansfont{Libertinus Sans}
\setmonofont{Libertinus Mono}
\usepackage{unicode-math}
\setmathfont{Libertinus Math}

\newcommand{\entry}[2]{\extramarks{#1}{#1}\textbf{#1} \ #2.}
\newcommand{\subentry}[2]{\indent\textbf{#1} \ #2.}

\usepackage{hyperref}


\begin{document}
\title{\uppercase{Англійсько-український\\словник}}
\author{tellnobody1}
\maketitle

\thispagestyle{empty}
\clearpage

\begin{center}
\section{\uppercase{Англійський алфавіт}}
\begin{tabular}{| c c c c c c c |}
  \hline
  Aa & Ee & Ii & Mm & Qq & Uu & Yy \\
  Bb & Ff & Jj & Nn & Rr & Vv & Zz \\
  Cc & Gg & Kk & Oo & Ss & Ww &    \\
  Dd & Hh & Ll & Pp & Tt & Xx &    \\
  \hline
\end{tabular}
\end{center}

\thispagestyle{empty}
\clearpage

\begin{twocolumn}

\section{@}
\entry{@}{равлик}

\section{A}
\entry{accessory}{причанд\'{а}л}\\
\entry{add-on}{додаток}

\section{B}
\entry{backpressure, back pressure}{проти\-т\'{и}ск, звор\'{о}тний тиск, контрт\'{и}ск}\\
\entry{browser}{переглядач}\\
\entry{byte}{беззн\'{а}кове ціле число, від 0 до $2^8$-1 включно}

\section{C}
\entry{caffeine}{каве\'{ї}н (-ну)}\\
\entry{compilation}{збірка, збирати}\\
\entry{compile}{збирати}\\
\entry{creator}{твор\'{е}ць}

\section{D}
\entry{default}{типовий}\\
\subentry{by \textasciitilde}{типово}\\
\entry{delivery}{дов\'{і}з (р. -в\'{о}зу)}\\
\entry{dislike}{невподоба}

\section{E}
\entry{easter egg}{яйц\'{е}-райц\'{е}, прихованка}

\section{F}
\entry{fantasy}{очар'я, фант\'{а}зія}\\
\entry{FOSS}{див. \hyperlink{foss}{free and open-source software}}\\
\hypertarget{foss}{\entry{free and open-source software}{вільне та відкрите програмне забезпечення}}

\section{G}
\entry{graceful shutdown}{ґр\'{е}чна пр\'{и}пинка}

\section{H}
\entry{handler}{\textbf{1.} ма\-ні\-пу\-л\'{я}\-тор; пр\'{и}ст\-рій керув\'{а}ння, керув\'{а}ч \textbf{2.} о\-пра\-ць\'{о}\-ву\-вач, прогр\'{а}ма о\-пра\-ць\'{о}\-ву\-ван\-ня}\\
\hypertarget{eventhandler}{\subentry{event \textasciitilde}{прогр\'{а}ма опраць\'{о}вування [опраць\'{о}вувач] под\'{і}й}}

\section{L}
\entry{listener}{\textbf{1.} слухач \textbf{2.} (в мовах Java, JavaScript) те саме, що \hyperlink{eventhandler}{event handler}}

\section{M}
\entry{menu}{\textbf{1.} спис страв, стравоспис, меню, карта (страв) \textbf{2.} (перен.) спис, меню}

\section{R}
\entry{review}{о́гляд}

\section{S}
\entry{setup}{підготування для роботи}\\
\hypertarget{startup}{\entry{start-up}{розпочин}}\\
\entry{startup}{див. \hyperlink{startup}{start-up}}

\section{T}
\entry{task}{завд\'{а}нн\'{я}}

\section{P}
\entry{path}{путь (ж. р.)}\\
\entry{problem}{за(в)д\'{а}ча}\\
\entry{project}{проєкт}

\end{twocolumn}

\end{document}
